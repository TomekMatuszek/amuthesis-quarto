% This is a LaTeX thesis template for Adam Mickiewicz University.
% to be used with Rmarkdown
% This template was produced by Jakub Nowosad
% Version: 16 February 2020

% Inspired by:
% This is a LaTeX thesis template for Monash University.
% to be used with Rmarkdown
% This template was produced by Rob Hyndman
% Version: 6 September 2016

\documentclass{amuthesis}
% \usepackage[polish]{babel}
\usepackage{polski}
\renewcommand{\figurename}{Figure} % Redefine default figure caption %
\renewcommand{\tablename}{Table} % Redefine default table caption %
%%%%%%%%%%%%%%%%%%%%%%%%%%%%%%%%%%%%%%%%%%%%%%%%%%%%%%%%%%%%%%%
% Add any LaTeX packages and other preamble here if required
%%%%%%%%%%%%%%%%%%%%%%%%%%%%%%%%%%%%%%%%%%%%%%%%%%%%%%%%%%%%%%%
\usepackage{booktabs,tabularx} % Allows kableExtra to work %
\usepackage{indentfirst} % Adds indent in the first paragraph %
\usepackage{bookmark} % Adds indent in the first paragraph %

\author{Tomasz Matuszek}
\title{Measuring impact of addition of Landsat 8 thermal band on
supervised land cover classification results}
\def\titleeng{Ocena wpływu zastosowania kanału termalnego Landsat na
wyniki nadzorowanej klasyfikacji pokrycia terenu}
\def\degreetitle{Praca inżynierska}
\def\major{Geoinformacja}
\def\albumid{455828}
\def\thesisyear{2022}
% Add subject and keywords below
\hypersetup{
     %pdfsubject={The Subject},
     %pdfkeywords={Some Keywords},
     pdfauthor={Tomasz Matuszek},
     pdftitle={Measuring impact of addition of Landsat 8 thermal band on
supervised land cover classification results},
     pdfproducer={quarto with LaTeX}
}

\bibliography{thesis,packages}

\begin{document}

\pagenumbering{arabic}

\titlepage

\bookmarksetup{startatroot}

\hypertarget{abstract}{%
\chapter*{Abstract}\label{abstract}}
\addcontentsline{toc}{chapter}{Abstract}

\textbf{Abstrakt}

Streszczenie powinno przedstawiać skrótowo główny problem pracy i jego
rozwiązanie. Możliwa struktura streszczenia to: (1) 1-3 zdania wstępu do
problemu (czym się zajmujemy, dlaczego jest to ważne, jakie są
problemy/luki do wypełnienia), (2) 1 zdanie opisujące cel pracy, (3) 1-3
zdania przedstawiające użyte materiały (dane) i metody (techniki,
narzędzia), (4) 1-3 zdania obrazujące główne wyniki pracy, (5) 1-2
zdania podsumowujące; możliwe jest też określenie dalszych
kroków/planów.

Słowa kluczowe: (4-6 słów/zwrotów opisujących treść pracy, które nie
wystąpiły w tytule)

\textbf{Abstract}

The abstract must be consistent with the above text.

Keywords: (as stated before)

\newpage

\setstretch{1.2}\sf\tighttoc\doublespacing

\bookmarksetup{startatroot}

\hypertarget{sec-intro}{%
\chapter{Introduction}\label{sec-intro}}

\begin{itemize}
\item
  applications and relevance of land cover maps
\item
  machine learning and supervised classification of satellite images as
  a tool for creating land cover maps
\item
  pointing out that thermal band if often omitted in land cover
  classification models, exact impact of thermal factor isn't fully
  clear
\item
  goal of the thesis is to create land cover map of Poznań metropolitan
  area and measure the impact of thermal band on the model results
\end{itemize}

\begin{center}\rule{0.5\linewidth}{0.5pt}\end{center}

Wprowadzenie powinno mieć charakter opisu od ogółu do szczegółu (np.
trzy-pięć paragrafów). Pierwszy paragraf powinien być najbardziej
ogólny, a kolejne powinny przybliżać czytelnika do problemu.
Przedostatni paragraf powinien określić jaki jest problem (są problemy),
który praca ma rozwiązać i dlaczego jest to (są one) ważne.

Wprowadzenie powinno być zakończone stwierdzeniem celu pracy. Dodatkowo
tutaj może znaleźć się również krótki opis co zostało zrealizowane w
pracy.

Pisząc ten rozdział proszę pomyśleć o osobach, które zupełnie nie znają
opisywanej tematyki. Należy tutaj krok po kroku wyjaśnić podstawowe
koncepcje, istotność problemu, wyniki poprzednich podobnych badań, itd.
Ten rozdział obejmuje tylko kwestie, które już zostały wykonane przez
inne osoby - nowe wyniki mają swoje miejsce w rozdziale
\textbf{?@sec-wyniki}.

Każda kwestia opisana w tym rozdziale powinna być cytowana. Dodatnie
cytowania odbywa się poprzez uzupełnienie pliku \texttt{thesis.bib}
zapisem w formacie BibTeX, a następnie dodanie nazwy referencji
poprzedzonej znakiem \texttt{@}. Przykładowo, zacytowanie książki
Geocomputation with R odbywa się poprzez
\autocite{lovelace_geocomputation_2019}.

W przypadku, gdy cytowanie zostało poprawnie wpisane oraz istnieje w
pliku \texttt{thesis.bib} to bibliografia powinna się automatycznie
wygenerować na końcu pracy.

W przypadku, gdy praca dyplomowa opisuje konkretny obszar to można po
tym rozdziale stworzyć kolejny rozdział opisujący ``obszar badań''.

Ten i kolejne rozdziału moją mieć także podrozdziały. Tworzenie
podrozdziałów polega na stworzeniu nowej linii rozpoczynającej się od
znaków \texttt{\#\#} a następnie tytułu podrozdziału. Dodatkowo w
postaci \texttt{\{\#sec-\}} można dodać skrót nazwy
rozdziału/podrozdziału umożliwiający odnoszenie się do niego używając
operatora \texttt{{[}-@sec{]}}.

\bookmarksetup{startatroot}

\hypertarget{sec-data}{%
\chapter{Source data}\label{sec-data}}

Satellite imagery used in our model was downloaded from Landsat ARD
dataset, provided by GLAD laboratory at the University of Maryland
\autocite{potapov2020}. Training points were obtained from LUCAS dataset
created by Eurostat \autocite{dandrimont2020}. This data was
pre-processed and then used to train the model and validate its
performance. Middle-West Poland was chosen as a training area for which
satellite imagery and land cover data were downloaded (Figure
\ref{fig-rycina1}).

\begin{figure}[t]

{\centering \includegraphics[width=1\textwidth,height=4.16667in]{./figures/study_area.png}

}

\caption{\label{fig-rycina1}Training area}

\end{figure}

\hypertarget{sec-sat}{%
\section{Satellite imagery}\label{sec-sat}}

Satellite imagery from GLAD Landsat ARD is available in 16-day interval
composites and is divided into 1 x 1 geographic degrees tiles.
Processing of original Landsat images performed by GLAD team included
conversion of spectral bands to top-of-atmosphere (TOA) reflectance,
conversion of thermal band to brightness temperature (BT) in Kelvins,
scaling the values of all bands as well as adding quality flag for every
pixel \autocite{potapov2020}.

Satellite images for eight 1 x 1 degree tiles (Figure \ref{fig-rycina1})
were downloaded using GLAD Tools v1.1 and PERL programming language.
These images come from 10th interval of the year 2018, so downloaded
mosaics consist of images created between 24.05.2018 and 8.06.2018. All
downloaded images were merged and reprojected from WGS84 coordinate
reference system (EPSG:4326) to UTM zone 33N (EPSG:32633). Every band
was also resampled to 30 meters resolution. In addition, four spectral
indices were derived: Normalized Difference Vegetation Index (NDVI),
Modified Normalized Difference Water Index (MNDWI), Normalized
Difference Moisture Index (NDMI) and Modified Bare soil Index (MBI).
Formulas used to calculate these indices can be found in below table
(Table \ref{tbl-tabela1}).

\hypertarget{tbl-tabela1}{}
\begin{table}
\caption{\label{tbl-tabela1}Formulas of spectral indices dervied from Landsat data }\tabularnewline

\centering
\begin{tabular}{>{\raggedleft\arraybackslash}p{2cm}>{\raggedright\arraybackslash}p{4cm}}
\toprule
a & b\\
\midrule
1 & a\\
2 & b\\
3 & c\\
4 & d\\
5 & e\\
\bottomrule
\end{tabular}
\end{table}

\hypertarget{sec-landcover}{%
\section{Land cover data}\label{sec-landcover}}

Data collected during LUCAS survey performed by Eurostat was chosen as
land cover training set. It seems to be the most accurate and
comprehensive dataset containing information about land use and land
cover \autocite{pflugmacher2019} due to the fact, that every point was
either manually photo-interpreted or assessed during \emph{in-situ}
visit.

LUCAS survey consists of two phases. First phase is based on grid of
points with 2km spacing covering whole territory of the European Union
(which equals to more than 1 million points). Each point of the grid is
visually interpreted using ortho-photos or satellite images, and
classified into one of seven major land-cover classes. These are: arable
land, permanent crops, grassland, wooded areas/shrub land, bare land,
artificial land and water. In the second phase a subsample of grid
points is selected and then visited by Eurostat surveyors. They classify
each point according to full LUCAS land cover and land use
classification. The survey takes place in the spring and summer in order
to observe chosen places in high vegetation season.

Surveyor not only assign a land cover and land use classes to a point,
but they also add auxillary information such as plant species present at
the site, percentage of land coverage for chosen class, height of the
trees and their maturity as well as information about water management
and irrigation. If there are more than one land cover/land use types at
the point, observer can also assign secondary class for every LUCAS
point. TUTAJ BĘDZIE REFERENCJA DO LUCAS COPERNICUS REPORT GDY DOWIEM
SIĘ, JAK GO CYTOWAĆ

Majority of the training points used to train our classification model
were points from the second phase of LUCAS survey, also called LUCAS
Micro Data. We downloaded a total of 4153 points for our study area.
Pre-processing step included omitting records with missing data,
excluding linear artificial land cover classes (e.g.~roads or railways)
and excluding points that were surveyed more than 500 meters from their
theoretical location. In the next step, detailed land cover classes were
aggregated into eight main groups of land cover types. Then, we filtered
some of the classes according to the percentage of land coverage or
percentage of impervious surface coverage (Table \ref{tbl-tabela2}).

\hypertarget{tbl-tabela2}{}
\begin{table}
\caption{\label{tbl-tabela2}Filters applied to certain land cover groups }\tabularnewline

\centering
\begin{tabular}{>{\raggedleft\arraybackslash}p{2cm}>{\raggedright\arraybackslash}p{4cm}}
\toprule
a & b\\
\midrule
1 & a\\
2 & b\\
3 & c\\
4 & d\\
5 & e\\
\bottomrule
\end{tabular}
\end{table}

For the least frequent classes in the LUCAS Micro Data dataset - bare
land, artificial land and water bodies - we also added points classified
during the first phase of LUCAS survey (Figure \ref{fig-rycina2}). This
step was necessary to ensure that every land cover class is represented
by enough number of points. It wasn't possible only for wetlands type,
because of lack of such category in the first phase classification. At
the end of the pre-processing, we were left with 3778 training points.

\begin{figure}[t]

{\centering \includegraphics[width=1\textwidth,height=4.16667in]{./figures/lucas_data.png}

}

\caption{\label{fig-rycina2}Distribution of land cover classes after
pre-processing}

\end{figure}

Later in the analysis, after extracting values from Landsat ARD raster,
LUCAS points were also filtered using quality flag provided. Only points
with clear-sky quality flag were taken into account during the process
of model training. We also excluded water bodies points in which NDWI
was lower than 0. These two conditions excluded over 400 points in
total.

\begin{figure}[t]

{\centering \includegraphics[width=1\textwidth,height=4.16667in]{./figures/lucas_distribution.png}

}

\caption{\label{fig-rycina3}Spatial distribution of LUCAS training
points after pre-processing}

\end{figure}

\bookmarksetup{startatroot}

\hypertarget{sec-methods}{%
\chapter{Methods}\label{sec-methods}}

\hypertarget{sec-ml}{%
\section{Machine learning}\label{sec-ml}}

\begin{itemize}
\item
  what is machine learning and what are its applications
\item
  classification vs regression algorithms
\item
  supervised and unsupervised classification
\end{itemize}

\hypertarget{sec-rf}{%
\subsection{Random forest algorithm}\label{sec-rf}}

\begin{itemize}
\item
  what is a decision tree
\item
  how random forest algorithm works
\end{itemize}

\hypertarget{sec-resampling}{%
\subsection{Model quality assessment}\label{sec-resampling}}

\begin{itemize}
\item
  idea of resampling
\item
  measures and indices of classification model quality
\end{itemize}

\hypertarget{sec-tuning}{%
\subsection{Parameter tuning}\label{sec-tuning}}

\begin{itemize}
\item
  what is tuning of model's parameters
\item
  idea of nested resampling
\end{itemize}

\hypertarget{sec-r}{%
\section{R language environment}\label{sec-r}}

Short description of R and RStudio environment. List of used libraries
and packages.

\begin{center}\rule{0.5\linewidth}{0.5pt}\end{center}

Rozdział \textbf{Metody} zawiera opis użytych metod (np. statystycznych
czy geostatystycznych) oraz technologii (np. pakiety R). Opis każdej z
metod czy technologi powinien być zwarty i zawierać tylko najważniejsze
informacje z punktu widzenia pracy dyplomowej.

Każda użyta metoda i technologia powinna być zacytowana. W przypadku
pakietów R, wystarczy wypełnić poniższy blok kodu (zwróć uwagę, że ten
blok kodu ma parametr \texttt{echo:\ false}; oznacza to, że będzie on
niewidoczny w wynikowym pliku PDF)\ldots{}

\ldots{} a następnie zacytować pakiet używając znaku \texttt{@}, po
którym podać nazwę pakietu rozpoczynającą się od prefiksu \texttt{R-}.
Przykładowe cytowanie języka R bez nawiasu to \textcite{R-base}, a
pakietu \textbf{kableExtra} w nawiasie to \autocite{R-kableExtra}.
Więcej przykładów cytowania można znaleźć na stronie
https://rmarkdown.rstudio.com/authoring\_bibliographies\_and\_citations.html\#citations.

W przypadkach, gdy cytowanie istnieje, ale nie jest pakietem R to należy
dodać je do pliku \texttt{thesis.bib} i użyć powyższej składni ze
znakiem \texttt{@}. W ostateczności, gdy dana technologia nie posiada
cytowania, należy podać jej adres internetowy.

\bookmarksetup{startatroot}

\hypertarget{sec-results-map}{%
\chapter{Result of the model - land cover map}\label{sec-results-map}}

\begin{itemize}
\item
  land cover map of Poznań metropolitan area
\item
  probability map of model results
\end{itemize}

\begin{center}\rule{0.5\linewidth}{0.5pt}\end{center}

Część \textbf{Wyniki} może składać się z jednego lub więcej rozdziałów.
Każdy z tych rozdziałów powinien mieć tytuł adekwatny do swojej treści.

Rozdziały wynikowe powinny korzystać z wiedzy opisanej w poprzednich
rozdziałach (Rozdziały \textbf{?@sec-lit}, \textbf{?@sec-dane},
\textbf{?@sec-metody}). W przypadku prac analitycznych, ich treść
powinna przedstawiać kolejne etapy eksploracji i analizy danych. W
przypadku prac technicznych, treść tych rozdziałów powinna opisywać
stworzone narzędzia, a następnie pokazywać ich zastosowanie/a.

W przypadku prac technicznych warto pokazywać fragmenty napisanego
rozwiązania lub jego wywołania używając bloków kodu.

\begin{Shaded}
\begin{Highlighting}[]
\NormalTok{moja\_funkcja }\OtherTok{=} \ControlFlowTok{function}\NormalTok{(x)\{}
  \FunctionTok{cat}\NormalTok{(x, }\StringTok{"rządzi!"}\NormalTok{)}
\NormalTok{\}}
\FunctionTok{moja\_funkcja}\NormalTok{(}\StringTok{"Autor tej pracy"}\NormalTok{)}
\end{Highlighting}
\end{Shaded}

\begin{verbatim}
Autor tej pracy rządzi!
\end{verbatim}

\bookmarksetup{startatroot}

\hypertarget{sec-results-eval}{%
\chapter{Assessing model quality}\label{sec-results-eval}}

Table with quality indices:

\begin{itemize}
\item
  overall accuracy (OA)
\item
  classification error (CE)
\item
  producer's and user's accuracy (PA, UA)
\item
  Kappa coefficient
\end{itemize}

\begin{center}\rule{0.5\linewidth}{0.5pt}\end{center}

Część \textbf{Wyniki} może składać się z jednego lub więcej rozdziałów.
Każdy z tych rozdziałów powinien mieć tytuł adekwatny do swojej treści.

Rozdziały wynikowe powinny korzystać z wiedzy opisanej w poprzednich
rozdziałach (Rozdziały \textbf{?@sec-lit}, \textbf{?@sec-dane},
\textbf{?@sec-metody}). W przypadku prac analitycznych, ich treść
powinna przedstawiać kolejne etapy eksploracji i analizy danych. W
przypadku prac technicznych, treść tych rozdziałów powinna opisywać
stworzone narzędzia, a następnie pokazywać ich zastosowanie/a.

W przypadku prac technicznych warto pokazywać fragmenty napisanego
rozwiązania lub jego wywołania używając bloków kodu.

\begin{Shaded}
\begin{Highlighting}[]
\NormalTok{moja\_funkcja }\OtherTok{=} \ControlFlowTok{function}\NormalTok{(x)\{}
  \FunctionTok{cat}\NormalTok{(x, }\StringTok{"rządzi!"}\NormalTok{)}
\NormalTok{\}}
\FunctionTok{moja\_funkcja}\NormalTok{(}\StringTok{"Autor tej pracy"}\NormalTok{)}
\end{Highlighting}
\end{Shaded}

\begin{verbatim}
Autor tej pracy rządzi!
\end{verbatim}

\bookmarksetup{startatroot}

\hypertarget{sec-results-therm}{%
\chapter{Evaluation of thermal band's impact on prediction
results}\label{sec-results-therm}}

\begin{itemize}
\item
  mean temperature for every predicted land cover class
\item
  variable importance plots, variable profiles
\item
  thermal band importance map (two methods: raster aggregation and
  interpolation of importance in LUCAS points)
\item
  mean importance of thermal band on each land cover class
\item
  difference raster map between prediction with and without thermal band
  included, transition matrix
\end{itemize}

\begin{center}\rule{0.5\linewidth}{0.5pt}\end{center}

Część \textbf{Wyniki} może składać się z jednego lub więcej rozdziałów.
Każdy z tych rozdziałów powinien mieć tytuł adekwatny do swojej treści.

Rozdziały wynikowe powinny korzystać z wiedzy opisanej w poprzednich
rozdziałach (Rozdziały \textbf{?@sec-lit}, \textbf{?@sec-dane},
\textbf{?@sec-metody}). W przypadku prac analitycznych, ich treść
powinna przedstawiać kolejne etapy eksploracji i analizy danych. W
przypadku prac technicznych, treść tych rozdziałów powinna opisywać
stworzone narzędzia, a następnie pokazywać ich zastosowanie/a.

W przypadku prac technicznych warto pokazywać fragmenty napisanego
rozwiązania lub jego wywołania używając bloków kodu.

\begin{Shaded}
\begin{Highlighting}[]
\NormalTok{moja\_funkcja }\OtherTok{=} \ControlFlowTok{function}\NormalTok{(x)\{}
  \FunctionTok{cat}\NormalTok{(x, }\StringTok{"rządzi!"}\NormalTok{)}
\NormalTok{\}}
\FunctionTok{moja\_funkcja}\NormalTok{(}\StringTok{"Autor tej pracy"}\NormalTok{)}
\end{Highlighting}
\end{Shaded}

\begin{verbatim}
Autor tej pracy rządzi!
\end{verbatim}

\bookmarksetup{startatroot}

\hypertarget{conclusion}{%
\chapter{Conclusion}\label{conclusion}}

\begin{itemize}
\item
  land cover map of Poznań metropolitan area was created, impact of
  thermal band on classification results was measured
\item
  despite thermal band having low overall impact on model results, there
  is a strong spatial auto-correlation for its importance
\item
  land surface temperature was especially significant for land cover
  classification of urban areas, it helped in identify built-up areas
\item
  it may mean that thermal band will become increasingly important in
  studies on urban sprawl and suburbanisation
\item
  better land cover maps will help in better management of metropolitan
  areas growth and quantifying impact of urbanisation on natural
  environment more precisely
\end{itemize}

\begin{center}\rule{0.5\linewidth}{0.5pt}\end{center}

Podsumowanie pracy jest w pewnym sensie znacznie rozbudowanym
abstraktem. Należy wyliczyć i opisać osiągnięcia uzyskane w pracy
dyplomowej. Tutaj jednak (w przeciwieństwie do np. rozdziału
\textbf{?@sec-wprowadzenie}) należy przechodzić od szczegółu do ogółu -
co zostało stworzone/określone, jak zostało to zrobione, jakie ma to
konsekwencje, itd.

Ten rozdział powinien też zawierać opis kwestii, których nie udało się
rozwiązać w pracy dyplomowej (i dlaczego się nie udało) oraz pomysły na
przyszłe ulepszenie uzyskanych wyników lub dalsze badania.

\printbibliography[heading=bibintoc, title=Bibliografia]

\end{document}
